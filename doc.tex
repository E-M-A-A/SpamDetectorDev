% This is a simple sample document.  For more complicated documents take a look in the exercise tab. Note that everything that comes after a % symbol is treated as comment and ignored when the code is compiled.

\documentclass{article} % \documentclass{} is the first command in any LaTeX code.  It is used to define what kind of document you are creating such as an article or a book, and begins the document preamble

\usepackage{amsmath} % \usepackage is a command that allows you to add functionality to your LaTeX code

\title{Storytelling} % Sets article title
\author{Emmanuele Virginio Coppola, Muriel Rossi, Alessandro Marigliano} % Sets authors name

% The preamble ends with the command \begin{document}   
\begin{document} % All begin commands must be paired with an end command somewhere 
    \maketitle % creates title using information in preamble (title, author, date)  $\downarrow$   \date{\today} % Sets date for date compiled

    
    \section{Introduzione} % creates a section
    
    La pubblicità invasiva si rivela essere sempre maggiormente un problema asfissiante all'interno del mondo dei socialnetwork. Nonostante siano state sviluppate diverse tecniche
    allo scopo di domare questa piaga, alcune inserzioni riescono tuttavia a sfuggirne, intasando interfacce dedicate all'informazione ed 
    all'intrattenimento, che subiscono così una svalutazione, venendo "soffocate".
    

    \textbf{Legge di Reed: il valore di una rete sociale è direttamente proporzionale ad una funzione esponenziale in N:}
    \begin{equation} % Creates an equation environment and is compiled as math
        V=a*N+b*N^2 + c*2^n
        \end{equation}
    

    La legge sopra citata, descrive come il valore di una rete, formata da interazioni ed intrecci, cresca esponenzialmente con la dimensione di quest'ultima. (1).
    In particolare, nel caso di Internet, si osserva come il suo valore tenda a crescere in modo esponenziale se associato a gruppi con interessi
     comuni, che condividono idee, interessi ed obiettivi.
    Da questo se ne deduce come invece, avvisi pubblicitari, spam ed altra "informazione spazzatura", possano far decadere il valore di una rete.
    
    Questo progetto si prefige lo scopo di manutenere una piattaforma libera dal bombardamento pubblicitario e dare modo all'utenza di potersi esprimere liberamente
    all'interno di questa, libera dall'ansia e dalla continua distrazione provocata dal chiasso delle inserzioni.
    \section{Descrizione dell'agente}
    Allo scopo di realizzare questo progetto, è stato introdotto un agente intelligente che, operando nel contesto dei commenti relativi ad ogni 
    pubblicazione effettuata sulla piattaforma, sarà in grado di segnalare gli elementi con sospetto di promotion.
    \subsection{Obiettivi}
    L'obiettivo dell'agente sarà quello di analizzare
    \subsection{Specifica PEAS}
    ftgh
    \subsection{Obiettivi}
    \section{Raccolta, analisi e preprocessing dei dati}
    \subsection{Scelta del dataset}
    \subsection{Pulizia del dataset}
    \subsection{Bilanciamento del dataset}




\end{document} % This is the end of the document